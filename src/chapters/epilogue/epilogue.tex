\noindent Οι μαγνητικά φορτισμένες μελανές οπές είναι ευσταθείς, και αποτελούν λύσεις του καθιερωμένου προτύπου και της βαρύτητας. Είναι ικανές να υποστηρίξουν ισχυρά μαγνητικά πεδία. Τα πεδία στην περιοχή κοντά στον ορίζοντα οδηγούν σε μη τετριμμένα κβαντικά φαινόμενα, που διαφοροποιούν τις μαγνητικά φορτισμένες μελανές οπές από τις αντίστοιχες αφόρτιστες μελανές οπές τύπου Schwarzschild.\\

Στην εξωτερική περιοχή κοντά στον ορίζοντα 
%των μαγνητικά φορτισμένων οπών αναμένονται αρκετά ισχυρά πεδία %έτσι ώστε 
η κατάσταση κενού του καθιερωμένου προτύπου είναι ασταθής. Η αστάθεια προκύπτει από την σύζευξη του σπιν των ηλεκτρικά φορτισμένων μποζονίων $W$ με το μαγνητικό πεδίο, 
%και οδηγεί στην αρνητική συνεισφορά του χαμηλότερου επιπέδου %Landau στην ενέργεια με συνέπεια τα μποζόνια να γίνονται 
καθιστώντας τα ταχυονικά. Η αστάθεια οδηγεί 
%διορθώνεται ελαχιστοποιώντας τη συνάρτηση δράσης του %ηλεκτρασθενούς (EW) μοντέλου χωρίς την προυπόθεση το πεδίο Higgs %να καταλαμβάνει κατάσταση με ασύμμετρη αναμενόμενη τιμή ως προς %τους SU(2)$\times$U(1) μετασχηματισμούς της EW θεωρίας. Τα πεδία %του μοντέλου θεωρούνται δυναμικά και παρουσιάζονται οι %αντίστοιχες εξισώσεις κίνησης, αποτέλεσμα της ελαχιστοποίησης. %Το κύριο σημείο είναι το 
στη δημιουργία συμπυκνωμάτων των μποζονίων  
%των εγκάρσιων (στο μαγνητικό πεδίο) συνιστωσών του μποζονικού %πεδίου 
$W$ στην εξωτερική περιοχή του ορίζοντα των φορτισμένων οπών, σπάζοντας τη σφαιρική συμμετρία. 
%Η μιγαδική φάση του συμπυκνώματος, είναι ασυνεχής σε σημεία %εντός κλειστών καμπυλών γύρω από μηδενισμούς του. Οι κλειστές %καμπύλες με ένα μηδενισμό σχηματίζουν κβαντικές δίνες με διαφορά %φάσης $\pm 2\pi$. 
Εντός του φλοιού που σχηματίζουν ο ορίζοντας και η ακτίνα $r\subscr{\scriptstyle W}$, ο μηχανισμός Higgs καταστέλλεται, τα φερμιονικά πεδία καθίστανται άμαζα και το κενό γίνεται υπεραγώγιμο.\\

Στο μη σχετικιστικό, σφαιρικό πρόβλημα Haldane (στην παρουσία μαγνητικού μονοπόλου) η χαμηλότερη στάθμη Landau εκδηλώνει
%του σφαιρικού προβλήματος με 
εκφυλισμό $Q+1$, όπου $Q$ το μαγνητικό φορτίο. Ο εκφυλισμός και οι ιδιοσυναρτήσεις \eqref{monomial eigenfunction} χρησιμοποιούνται για την επίλυση του σχετικιστικού προβλήματος. Οι λύσεις της γενικής εξίσωσης Dirac (ενότητα \ref{sec:dirac solutions in magnetic mon field}) διαχωρίζουν τα φερμιονικά πεδία σε εξερχόμενα, που μπορούν να διαφύγουν από το βαρυτικό πεδίο της φορτισμένης μελανής οπής, και εισερχόμενα. Η ταξινόμηση επιτυγχάνεται στη βάση του κανόνα \eqref{rule for out in fields}, του υπερφορτίου και της τετραδιάστατης χειραλικότητας των φερμιονικών πεδίων.
%, τα πεδία που μπορούν να διαφύγουν. 
Ο εκφυλισμός κάθε φερμιονικής αναπαράστασης της $SU(3)\times SU(2)\times U(1)$ (πίνακας \ref{tab:table with particles and su3, su2 representations}) παρουσιάζεται στον πίνακα \ref{table: out-in fields and degeneracy for each representation}. Ο συνολικός εκφυλισμός των εξερχόμενων φερμιονίων ισούται με $9Q$.
%, όπου Q το μαγνητικό φορτίο της οπής. 
Τα $9Q$ πεδία αυτά είναι άμαζα εντός της ηλεκτρασθενούς κορώνας και συνεισφέρουν στην ακτινοβολία Hawking. Ο ρυθμός εκπομπής της ενέργειας 
%μέσω της αναβαθμισμένης ακτινοβολίας Hawking, 
υπολογίζεται στην ενότητα \ref{sec:energy radiation of one dimensional fermionic model} και είναι ανάλογος του μαγνητικού φορτίου ($\sim QT\superscr{2}$). Συγκρίνοντας με τον ρυθμό εξαύλωσης μιας μελανής οπής Schwarzschild 
%οπής ($\sim T\superscr{2}$) 
εκδηλώνεται ενίσχυση του φαινομένου της ακτινοβολίας Hawking. Μια μη ακραία μαγνητικά φορτισμένη οπή, $GM^2 > \pi Q^2/e^2$, εξαϋλώνεται πολύ πιο γρήγορα 
%(ανάλογο του Q) 
από μια αφόρτιστη οπή της ίδιας μάζας. Η φορτισμένη οπή θα σταματήσει να ακτινοβολεί όταν η μάζα ελαττωθεί στην οριακή τιμή $GM^2 = \pi Q^2/e^2$.