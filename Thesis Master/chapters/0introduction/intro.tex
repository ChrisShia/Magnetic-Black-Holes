%Η διατριβή αναφέρεται στην γενίκευση των Schwarzschild μελανών οπών με την ένταξη μαγνητικού μονοπολικού φορτίου. 
\par Η σύζευξη βαρύτητας και ηλεκτρομαγνητισμού
%, εδραιώνεται με τη γενίκευση της γεωμετρίας Schwarzschild. Η %λύση, για σφαιρικώς συμμετρικά σώματα, οδηγεί στη μετρική της  Η %γεωμετρία 
προβλέπει την ύπαρξη φορτισμένων μελανών οπών που περιγράφονται από τη γεωμετρίας Reissner–Nordström.  
%ανωμαλίες με 
Οι φορτισμένες μελανές οπές έχουν κοινά χαραχτηριστικά με τις αφόρτιστες μελανές οπές Schwarzschild 
%μελανών οπών, 
όπως, ορίζοντα, μη μηδενική θερμοκρασία και εντροπία, και εξαϋλώνονται μέσω εκπομπής ακτινοβολίας. 
%η μελανή οπή. 
Σημαντική διαφορά με τις αφόρτιστες μαύρες τρύπες 
%οπές χωρίς φορτίο 
είναι το γεγονός ότι η εξαΰλωση παύει να υφίσταται 
%της εξάτμισης τους, 
όταν η μάζα και το φορτίο ικανοποιούν την ισότητα $GM\superscr{2}=\frac{\pi Q\superscr{2}}{e\superscr{2}}$. Σε αυτή την οριακή περίπτωση, η θερμοκρασία Hawking μηδενίζεται. \\

\par Η διατριβή επικεντρώνεται στη μελέτη μαγνητικά φορτισμένων μαύρων οπών \cite{Maldacena_2021}, λόγω της ευστάθειας τους σε σχέση με τις ηλεκτρικά φορτισμένες. Η μεγάλη μάζα των μαγνητικών μονοπόλων, καταστέλλει τη δημιουργία ζευγών σωματιδίων με θετικό και αρνητικό μαγνητικό φορτίο εντός σχετικά ισχυρών μαγνητικών πεδίων. Σε αντίθεση, η ευστάθεια των ηλεκτρικά φορτισμένων σωμάτων περιορίζεται από το κατώφλι $E\sim 2m\subscr{e}\superscr{2}$ ($\sim 10\superscr{18}V/m$), που προβλέπεται από τη δίδυμο γένεση Schwinger. Κοντά στον ορίζοντα μιας φορτισμένης μελανής οπής ($r\sim MG$), η τιμή του πεδίου είναι $\frac{Q}{M\superscr{2}G\superscr{2}}$. Ηλεκτρικά φορτισμένες οπές με φορτίο και μάζα που ικανοποιούν την ανισότητα $Q\,{\scriptscriptstyle\gtrsim}\,m\subscr{e}\superscr{2}M\superscr{2}G\superscr{2}$
αποφορτίζονται από τα σωματίδια με αντίθετο ηλεκτρικό φορτίο, που δημιουργούνται κατά τη δίδυμο γένεση. Η αποφόρτιση επιτυγχάνεται σε σχετικά σύντομο χρονικό διάστημα, καθιστώντας τις μη ευσταθείς.\\

\par Η ευστάθεια των μαγνητικών πεδίων όμως, δεν παραμένει χωρίς κατώφλι. Για συγκεκριμένες τιμές του μαγνητικού φορτίου και της μάζας δημιουργούν στην εξωτερική περιοχή του ορίζοντα μαγνητικό πεδίο με ένταση ίση με τη μάζα στο τετράγωνο των μποζονίων $W$ της ασθενούς αλληλεπίδρασης. Στην παρουσία τέτοιων μαγνητικών πεδίων λαμβάνουν χώρα διάφορα κβαντικά φαινομένα που οδηγούν σε μια νέα ευσταθή κατάσταση με μη μηδενικό μαγνητικό πεδίο. Τα φαινόμενα εκδηλώνονται σε ακτίνα της τάξης $r\,{\scriptstyle\sim}\,m\subscr{w}\superscr{-1}$ και ενισχύονται εντός της "ηλεκτρασθενούς κορώνας", $m\subscr{h}\superscr{-1}\,{\scriptscriptstyle<}\,r\,{\scriptscriptstyle<}\,m\subscr{w}\superscr{-1}$. Σε ακτίνα $r\,{\scriptstyle\sim}\,m\subscr{h}\superscr{-1}\,{\scriptscriptstyle<}\,m\subscr{w}\superscr{-1}\,$, όπου $h$ to σωματίδιο Higgs, η μη αβελιανή $SU(2)$ συνιστώσα του μαγνητικού πεδίου θωρακίζεται πλήρως, και μένει μόνο η συνιστώσα του αβελιανού $U(1)\subscr{Y}$ υπερφορτίου. 
%πεδίο φτάνει στην ευσταθή κατάσταση. Για μικρότερες ακτίνες το %πεδίο δεν είναι πλέον ο γραμμικός συνδυασμός των συνιστωσών %SU(2) και , αλλά γίνεται ένα εξολοκλήρου υπερμαγνητικό αβελιανό %πεδίο U(1)$\subscr{Y}$. 
Αυτό οφείλεται στη συμπύκνωση ηλεκτρικά φορτισμένων μποζονίων $W$. 
%γεννιούνται κατά τις κβαντικές διακυμάνσεις του κενού, και %αλληλεπιδρούν με το συμβατικό μαγνητικό πεδίο U(1)$\subscr{EM}$. %Η αλληλεπίδραση, οδηγεί σε αστάθειες των μποζονικών πεδίων και %προκαλεί τη δημιουργία συμπυκνώματος W. 
Οι Ambjorn και Olesen βρίσκουν (προσεγγιστικές) λύσεις που περιγράφουν το συμπύκνωμα των μποζονίων $W$, ως ένα περιοδικό πλέγμα κβαντικών δινών στο κάθετο στο μαγνητικό πεδίο επίπεδο \cite{AMBJORN1990193}. 
%διευθύνσεις ως προς το μαγνητικό πεδίο. 
Ένα τέτοιο συμπύκνωμα δημιουργείται στην εξωτερική περιοχή του ορίζοντα μιας σφαιρικά συμμετρικής μαγνητικής μελανής οπής. 
Το συμπύκνωμα αυτό ελαττώνει την ένταση της $SU(2)$ συνιστώσασ του μαγνητικού πεδίου μέχρι την πλήρη θωράκιση του.\\
%μέχρι η ένταση του να φτάσει σε τάξη μεγέθους %${\scriptstyle\sim\,\,m\superscr{2}\subscr{h}/e}$, όπου και %θωρακίζεται εντελώς. 

Το υπερμαγνητικό πεδίο που παραμένει, συζεύγνυται με το υπερφορτίο του Higgs και συνεισφέρει θετικά στην ενεργό μάζα του βαθμωτού πεδίου. Η θετική αυτή συνεισφορά 
%ορίζεται από τα επίπεδα Landau, είναι ανάλογη της έντασης του %υπερπεδίου και 
αναιρεί τη συνεισφορά του αρνητικού όρου μάζας στο δυναμικό του πεδίου Higgs. Η αναίρεση αυτή, καθιστά τη μηδενική τιμή του Higgs ως το μόνο ευσταθές ακρότατο του νέου ενεργού δυναμικού. Στην κατάσταση αυτή, το Higgs έχει μηδενική αναμενόμενη τιμή, η οποία είναι συμμετρική ως προς τους ηλεκτρασθενείς μετασχηματισμούς βαθμίδας $SU(2)\times U(1)\subscr{Y}$, και έτσι παύει η ρήξη της ηλεκτρασθενούς συμμετρίας.\\

\par Η επαναφορά της συμμετρίας εντός της ηλεκτρασθενούς κορώνας επιφέρει την καταστολή του μηχανισμού Higgs. Τα φερμιονικά πεδία καθιστώνται άμαζα και συνεισφέρουν στην ακτινοβολία Hawking. Οι λύσεις της βαρυτικής εξίσωσης Dirac εκδηλώνουν 
%τα απορρέοντα από την οπή, με συνολικό 
εκφυλισμό που είναι ανάλογος του μαγνητικού φορτίου $Q$ της μελανής οπής. Ο εκφυλισμός υπολογίζεται με βάση την ανάλυση του Haldane για το ανάλογο πρόβλημα Landau μη σχετικιστικών σωματιδίων σε σφαιρικά συμμετρικό μονοπολικό μαγνητικό πεδίο. Η ισχύς της ακτινοβολίας Hawking είναι ανάλογη του μαγνητικού φορτίου Q${\scriptstyle>>}$1. Επομένως αποτελεί ένα δραματικά αυξημένο ρυθμό εξάτμισης. 

